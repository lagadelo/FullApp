\documentclass[a4paper,11pt]{article}
\usepackage[utf8]{inputenc}
\usepackage[T1]{fontenc}
\usepackage{graphicx}
\usepackage{hyperref}
\usepackage{geometry}
\usepackage{xcolor}
\usepackage{listings}
\usepackage{amsmath}

\geometry{margin=2.5cm}

\title{Système de Simulation et d'Évaluation de Flottes de Drones}
\author{Projet Smalltalk / Pharo 13}
\date{\today}

\begin{document}
\maketitle

\tableofcontents

\section{Introduction}

La multiplication des systèmes autonomes, et notamment des drones, transforme profondément
les doctrines militaires modernes. Dans un contexte de \textbf{guerre de haute intensité},
les forces doivent affronter des adversaires disposant de moyens équivalents ou supérieurs,
où la masse, la coordination et la vitesse d'action jouent un rôle déterminant.  

La simulation de telles situations représente un outil stratégique : elle permet d'explorer
des scénarios variés, de tester des stratégies, et d'évaluer la robustesse de doctrines
opérationnelles avant leur mise en œuvre réelle.  

Le présent travail propose un environnement modulaire et extensible, développé en Pharo 13,
basé sur les traits et intégrant à la fois la modélisation de flottes de drones alliées et
hostiles, l'affectation de missions, la gestion de comportements et l'évaluation de stratégies
heuristiques. L'objectif est double : \textbf{tester la viabilité opérationnelle} et
\textbf{aider à la prise de décision} par simulation.

\section{Spécifications Fonctionnelles}
\begin{itemize}
  \item Gestion de drones avec attributs réels (latitude, longitude, altitude, carburant...).
  \item Support multi-flottes : alliées et hostiles.
  \item Définition de comportements : patrouille, retour à la base, aller en aide, statique.
  \item Missions de flotte : surveillance et sécurisation de zones polygonales.
  \item Détection d'ennemis, communication avec consensus.
  \item Gestion temps-réel configurable, exact si nécessaire.
\end{itemize}

\section{Architecture du Système}
\subsection{Traits}
\begin{itemize}
  \item \texttt{TLocatable} : gère la position (lat/lon/alt).
  \item \texttt{TEnergyConsumer} : gère carburant et consommation.
  \item \texttt{TSensorCarrier} : gère la perception et détection.
  \item \texttt{TCombatUnit} : gère les armes et attaques.
\end{itemize}

\subsection{Classes Principales}
\begin{itemize}
  \item \texttt{Drone} : unité de base, compose les traits.
  \item \texttt{DroneBehavior} et sous-classes (Patrol, ReturnToBase...).
  \item \texttt{DroneCluster} : gère un ensemble de drones.
  \item \texttt{ClusterMission} : assignation de missions.
  \item \texttt{SimulationState} : état global et moteur de simulation.
  \item \texttt{DroneMapView} : interface graphique (Roassal3 ou PharoOWS).
\end{itemize}

\begin{figure}[h!]
\centering
\fbox{\includegraphics[width=0.7\linewidth]{architecture.png}}
\caption{Diagramme d’architecture simplifié du système}
\end{figure}

\section{Évaluation de Stratégies}
\subsection{Stratégies Heuristiques Implémentées}
\begin{itemize}
  \item \textbf{ProactiveInterceptionStrategy}
  \item \textbf{AdaptiveEnergyStrategy}
  \item \textbf{DivideAndSurroundStrategy}
  \item \textbf{PriorityDefenseStrategy}
  \item \textbf{LearningBasedStrategy}
\end{itemize}

\subsection{Comparaison et Visualisation}
Chaque stratégie est évaluée selon :
\begin{itemize}
  \item Nombre de drones alliés survivants.
  \item Nombre de drones hostiles neutralisés.
  \item Consommation énergétique moyenne.
  \item Missions accomplies.
\end{itemize}

\begin{figure}[h!]
\centering
\fbox{\includegraphics[width=0.8\linewidth]{strategies.png}}
\caption{Exemple de graphe comparatif des stratégies exporté en PNG}
\end{figure}

\section{Exports Disponibles}
\begin{itemize}
  \item \textbf{CSV} : résultats numériques tabulés.
  \item \textbf{PNG} : graphes comparatifs.
  \item \textbf{JSON} : résultats exploitables en Python/JS.
  \item \textbf{LOG} : événements de simulation.
  \item \textbf{HTML (Chart.js)} : visualisation interactive dans le navigateur.
\end{itemize}

\section{Intégration avec PharoOWS}
PharoOWS peut être utilisé comme fond de carte pour visualiser la simulation
dans un contexte géographique réel (centré sur l’Europe).  
En absence de PharoOWS, le système reste fonctionnel avec Roassal3.

\begin{figure}[h!]
\centering
\fbox{\includegraphics[width=0.7\linewidth]{map_example.png}}
\caption{Exemple d’intégration avec PharoOWS pour fond de carte}
\end{figure}

\section{Application à la Guerre de Haute Intensité}

Dans un contexte de \textbf{guerre de haute intensité}, les opérations impliquent de multiples systèmes
autonomes, opérant dans des environnements fortement contestés et dynamiques.  

La démarche présentée ici trouve une utilité directe dans plusieurs dimensions :

\begin{itemize}
  \item \textbf{Planification et entraînement} : la possibilité de simuler différents scénarios
  avec plusieurs flottes hostiles permet aux états-majors de tester des doctrines, évaluer la robustesse
  de leurs tactiques et identifier les vulnérabilités potentielles avant un engagement réel.

  \item \textbf{Évaluation de missions complexes} : la gestion simultanée de missions (patrouille,
  sécurisation, interception) permet de reproduire les tensions d'une opération de haute intensité,
  où les ressources sont limitées et doivent être allouées intelligemment.

  \item \textbf{Comparaison de stratégies} : l’intégration d’heuristiques variées (interception proactive,
  défense prioritaire, adaptation énergétique, apprentissage) fournit une base expérimentale
  pour comparer l’efficacité de différentes approches face à des attaques massives de drones hostiles.

  \item \textbf{Résilience et coordination} : la simulation multi-flottes met en lumière les
  capacités de coopération inter-drones, de communication par consensus et de résilience
  en cas de perte de nœuds dans le réseau, des problématiques centrales dans la guerre moderne.

  \item \textbf{Décision assistée par simulation} : l’export automatique (CSV, JSON, HTML)
  facilite l’intégration des résultats dans des systèmes tiers (outils d’IA, plateformes C2),
  permettant de nourrir en temps réel la boucle décisionnelle.
\end{itemize}

\begin{figure}[h!]
\centering
\fbox{\includegraphics[width=0.8\linewidth]{wargame_scenario.png}}
\caption{Exemple de scénario simulé avec confrontation de flottes alliées et hostiles}
\end{figure}

\section{Conclusion}
L'environnement développé constitue une plateforme complète de simulation et d'évaluation
des stratégies de gestion de flottes de drones.  

Ses points forts sont :
\begin{itemize}
  \item la modularité de l'architecture (traits, comportements, missions, stratégies),
  \item la prise en charge du multi-flottes avec drones alliés et hostiles,
  \item l’intégration d’un moteur de simulation configurable en temps réel ou exact,
  \item les exports multiples (CSV, PNG, JSON, LOG, HTML) pour l’analyse et l’intégration
  dans des outils tiers,
  \item la compatibilité optionnelle avec PharoOWS pour un ancrage géographique réel.
\end{itemize}

\textbf{Le bénéfice majeur de cet environnement est de fournir un laboratoire virtuel
permettant d'expérimenter, comparer et affiner des stratégies dans un contexte réaliste
de guerre de haute intensité.}  

Il sert ainsi de support à la recherche opérationnelle, à l’entraînement et à la préparation
des forces, tout en réduisant les coûts et les risques associés aux expérimentations sur le terrain.

\end{document}
